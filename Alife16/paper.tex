\documentclass[letterpaper]{article}
\usepackage{natbib,alifeconf}
\usepackage[breaklinks=true]{hyperref}
\usepackage{url}

\graphicspath{{../img/}}
\DeclareGraphicsExtensions{.pdf}

\title{The human in the loop: volunteer-based metacomputers as a socio-technical
  system}
\author{Juan-J.~Merelo*$^1$, Paloma de las Cuevas $^1$, Pablo
  Garc\'ia-S\'anchez $^1$, Mario Garc\'ia-Valdez$^2$\\
\mbox{}\\
$^1$Dept. of Computer Architecture and Technology and CITIC University of Granada \\
$^2$Dept. of Graduate Studies at Instituto Tecnol\'ogico de Tijuana \\
{\tt jmerelo@ugr.es}, {\tt mario@tectijuana.edu.mx}}

\begin{document}
\maketitle

\begin{abstract}
Volunteer computing is a form of distributed computing where users
decide on their participation and the amount of time and other
resources they will ``lend''. This makes them an essential part of the
algorithm and of the performance of the whole system. As a
sociotechnical system, this participation follows some patterns and in
this paper we examine the result of several volunteer distributed
evolutionary computation experiments and try to find out what
those patterns are and what makes an experiment successful or not,
including the feedback loop that is created between the users and the
algorithm itself.
\end{abstract}

\section{Introduction}
\label{introduction}

Some time ago, our group faced the problem of diminishing funds for
buying new hardware. This was aggravated by the increasing maintenance
costs and extended downtime resulting from the continuous failures of
existing clusters.  Considering this, we leveraged our experience in
the design of web applications with JavaScript and other volunteer and
unconventional distributed evolutionary computing systems to design
and release a new free framework that would allow anyone to create a
volunteer distributed evolutionary computation (EC) experiment using
cloud resources as servers and browsers as clients. This framework was
called NodIO \citep{2016arXiv160101607M}. NodIO provides server
infrastructure for volunteer-based distributed evolutionary computing
experiments by providing a chromosome {\em pool}. This pool is used by
clients in browsers and any other using the application programming
interface (API) to put chromosomes and retrieve them, working then as
a loose, asynchronous and {\em ad hoc} connection among all clients
using it. 

This loose connection provides a low-overhead way to connect desktop
experiments with volunteer-based ones, with all of them contributing
to the pool, but every one of them working as separate island carrying
out their own evolutionary algorithms. This is why NodIO is proposed
mainly as a {\em complement} to existing resources such as desktop
systems or laptops. As long as it provides a non-null computational
capability that can help existing resources find the solution faster
it will have found its purpose. The main use case is someone setting
up NodIO in the cloud, writing a fitness evaluation function and
running a client from his or her own computer, but requesting help in
social networks for additional resources. That is why, in this system,
we have considered the whole {\em social} aspect in the design, with
issues related to security, trust and privacy among others. The
computing system becomes a {\em sociotechnical system}
\citep{vespignani2009predicting}. In this paper, we are going to focus
on measuring the response of users to experiments, that is, the time
they spend running it, but at the same time we will also focus on the
technical aspects of the server and how these might change the
behavior of users, improving the capability of the system.

Our research group is committed to open science, and we think this is
a very important part of the techno-social system. By being
transparent, incentives to cheat are reduced and, in fact, we have
detected no issue for the time being. Next we present the state of the
art in web-based volunteer computing systems along with attempts to
predict and model its behavior. 


%---------------------------------------------------------------
\section{State of the art}
\label{sec:soa}

Volunteer computing involves a user running a program voluntarily
and, as such, has been deployed in many different ways from the
beginning of the Internet, starting with the SETI@home framework for
processing extraterrestrial signals \citep{david-seti:home}. However
the dual introduction of JavaScript as a universal language for the
browser and the browser as an ubiquitous web and Internet client has
made this combination the most popular for volunteer computing
frameworks such as the one we are using here, and whose first version
was described in \citep{DBLP:conf/gecco/GuervosG15}.

JavaScript can be used for either unwitting
\citep{unwitting-ec,boldrin2007distributed} or volunteer
\citep{langdon:2005:metas,gecco07:workshop:dcor} distributed
evolutionary computation and it has been used ever since by several
authors, including more recent efforts
\citep{Desell:2008:AHG:1389095.1389273,duda2013distributed,DBLP:journals/corr/abs-0801-1210}. Many
other researchers have used Java \citep{chong:1999:jDGPi}; others have
embraced peer to peer systems
\citep{jin2006constructing,10.1109/ICICSE.2008.99,DBLP:conf/3pgcic/GuervosMFEL12}. These
computing platforms avoid single points of failure, the server, and
once installed need no effort to gather new users for an experiment,
but the cost of acquiring new users is high since they need to be set
up.

The number of users is key in the performance of these systems, but it
also essential to adapt
the algorithm itself to the available resources, as shown by
\citep{milani2004online}, although EAs can be readily distributed via
population splitting or by farming out the evaluation to all 
the nodes available. However, user churn affects experiment performance 
\citep{gonzalez2010characterizing,nogueras2015studying} and also the
performance of the algorithm itself
\citep{DBLP:journals/gpem/LaredoBGVAGF14}. All these issues imply that
a the performance of a volunteer system cannot be
measured without first understanding its dynamics. Initial
work was done for peer to peer systems by Stutzbach et
al. \citep{stutzbach2006understanding} and extended to volunteer
computing by Laredo et al. \citep{churn08,laredo2008rcp}. A similar
study was performed by Martínez et al. on the Capataz system
\citep{martinez2015capataz}; however, in this case the number of
computers used was known in advance and the main focus was on
measuring the speed up and how job bundling helped to reduce overhead and
enhance performance. On the contrary, in this paper, we will use {\em
 actual} volunteers.

Some of the essential metrics in volunteer computing like the
number of users or the time spent by every one in the
computation in browser-based volunteer computing experiments, have
only been studied in a limited way in 
\citep{DBLP:journals/gpem/LaredoBGVAGF14} on the basis of a single
run. Studies using volunteer computing platforms such as SETI@home
\citep{javadi2009mining,agajaj} found out that the Weibull, log-normal and
Gamma distribution 
modeled quite well the availability of resources in several clusters
of that framework; the shape of those distributions is a skewed bell
with more resources in the {\em low} areas than in the high areas:
there are many users that give a small amount of cycles, while there
are just a few that give many cycles. 

As far as we know, this paper presents one of the few experiments
that measure the performance of a sociotechnical
metacomputer, that is, a spontaneously created parallel computer that
uses social networks for operations such as gathering new
users. Apol{\'o}nia et al. \citep{apolonia2012enhancing} used the 
Facebook protocol to distribute tasks among the {\em walls} of
friends, explicitly using the social network for computing. However,
it stopped short of relating performance to the macro measures of the
users' social networks. As in the previous example, a social network
was used to get new network nodes; in the previous case a web page was
used, while Facebook's wall was used here.

In our case, social networks are an integral part of the system and
used to spontaneously obtain users. The algorithms used, as well as the methodology 
for gathering resources will be described next, 
together with the results obtained in this initial setup.


\section{Description of the framework}
\label{sec:description}

In general, a distributed volunteer-based evolutionary computation
system based on the browser is simply a client-server system
whose client is, or it can be, embedded in the browser via
JavaScript. Since JavaScript is  the only language that is present
across all browsers, the choice was quite clear. We should emphasize
that NodIO is more intended as an auxiliary computing engine, more
than the main one, so performance of JavaScript as a language is not
so important; even so, we have made a comparison between JavaScript and other languages
\citep{2015arXiv151101088M} that shows that the performance of JavaScript
is comparable to other interpreted languages; compiled languages would
be faster, but, of course, it is impossible to gather volunteers
spontaneously and without any installation with them.

In this sense, in this paper we propose the {\sf NodIO} framework, a
cloud or bare metal based volunteer evolutionary computing system
derived from the {\sf NodEO} library, whose architecture has been
developed using JavaScript on the client as well as the server.
All parts of the framework are free and available with a free license
from \url{https://github.com/JJ/splash-volunteer}.

Thus, {\sf NodIO} architecture has two tiers:\begin{enumerate}
\item A REST server, that is, a server that includes several {\em
  routes} 
  that can be called for storing and retrieving information (the `CRUD'' cycle:
  create, request, update, and delete) from the server. 
  A JSON data format is used for the communication between 
  clients and the server. There are two kinds of information:
  {\em problem} based, that is, related to the
  evolutionary algorithm such as {\tt PUT}ing a chromosome in or {\tt
  GET}ing a random chromosome from it, and {\em information} related
  to the performance and state of the experiments. It also performs logging
  duties, but they are basically a very lightweight and high performance
  data storage \citep{jj:idc:lowcost}.
  The server has the capability to
  run a single experiment, storing the chromosomes in a key-value store
  that is reset when the solution is found. This store can
  hold every chromosome in a particular experiment, or have a finite
  size that erases the oldest chromosomes once it has filled to
  capacity, acting as a cache. In this paper we will test both implementations.  
\item A client that includes the evolutionary algorithm as
  JavaScript code embedded in a web page that displays graphs, some
  additional links, and information on the experiment. This code runs
  an evolutionary algorithm {\em island} that starts with a random
  population, then after every 100 generations, it sends the best individual
  back to the server (via a {\tt PUT} request), and then requests a random
  individual back from the server (via a {\tt GET} request). We have
  kept the number of generations between migrations fixed since it is
  a way of finding out how much real work every client has done. 
\end{enumerate}

Figure \ref{fig:system} describes the general system architecture and
algorithm behavior. Different web technologies, such as JQuery and {\tt
  Chart.js} have been used to build the user interface elements of the
framework, a part of which is shown in Figure \ref{fig:screenshot} and
should be running in \url{http://nodio-jmerelo.rhcloud.com}.

\begin{figure}[!t]
\centering
\includegraphics[width=0.95\linewidth]{nodio-screenshot.png}
\caption{Screenshot showing the fitness and number of users as seen by
volunteers. The one on the left shows fitness, on the right the
accumulated number of users. \label{fig:screenshot}}
\end{figure}
%
\begin{figure}[!t]
\centering
\includegraphics[width=3in]{system.pdf}
\caption{Description of the proposed system. Clients execute a JavaScript EA
  in the browser, which, every 100 generations, sends the best
  individual and receives a random one back from the server.}
\label{fig:system}
\end{figure}

JavaScript is a functional language, so in order to work with a
problem, a fitness function must be supplied at the creation of the algorithm object, 
called {\tt Classic}. In this case the classical Trap 
function \citep{Ackley1987} has been used. Depending on the
problem additional handing functions and configurations 
can be supplied. 

The next Section will describe experiments performed to
establish a baseline performance and gather initial performance
results. In the first set of experiments, performed last year, we used the 40-trap
problem, while the current experiments changed to the more difficult 50-trap
in order to compare the performance with a more computationally intensive problem.

%---------------------------------------------------------------
\section{Modeling the performance of a volunteer-based distributed computer} 
\label{sec:experiments}

\begin{table}[htb]
\caption{Experiment table, with summary of results. \label{tab:runs}}
\begin{center}
\begin{tabular}{l|rrl}
\hline
Experiment & \#Runs & Different IPs & Traps \\
\hline
April 4th 4/4 & 57 & 191 & 40 \\
April 24th 4/24 &  231 & 559& 40  \\
July 31th 7/31 & 97 & 179 & 40 \\
\hline
February, cache=128 & 61 & 75 & 50  \\
February, cache=64 & 61 & 220 & 50  \\
February, cache=32 & 39 & 86 & 50  \\
% grep start ~/Code/splash-volunteer/data/2016/nodio-2016-2-28-cache=32.log
\hline
\end{tabular}
\end{center}
\end{table}
%
Initial experiments were set up using the OpenShift
PaaS \url{www.openshift.com}, which provides a free tier, making the whole experiment cost
equal to \$0.00. Experiments were
announced through a series of posts on Twitter and, in the latest case, Telegram, and
results were published in\citep{DBLP:conf/gecco/GuervosG15}. For the
purpose of this paper, we repeated the announcement several times
through the month of February of the next year. All  
in all, we have the set of runs with the characteristics shown in
Table \ref{tab:runs}. In general, every experiment took several
days. No particular care was taken about the time of the announcement
%Pablo: but it would great at least describe how many times was
%announced and how long
% We don't really tracked this... - JJ
or the particular wording. Every {\em experiment} consisted in running
until the solution of the 50-trap problem was found. When the correct
solution was sent to the server, the counter was updated and the pool
of solutions resets to the void set. There was no special intention to wait
until all clients had finished, thus it might happen that. In fact,
the islands running in the browser {\em spill} from one experiment to
the next. 
If an individual that is close or even at optimum value is stored in the cache 
the next experiment will be affected, and it will require less time 
to finish. This problem has been addressed in later versions of the
system, however, time to solution is not the important measure here,
where we are concerned with the computational 
power of the system established by the number of volunteers. 

The table \ref{tab:runs}shows that every experiment included more than 30 runs. The
number of different IPs intervening in them varied from more than one
hundred to more than five hundred in the second experiment, with a
number around 50 in the second, and most recent, batch of experiments. 
%
\begin{table}[htb]
\caption{Summary of time per run, number of IPs and number of {\tt
    PUT}s per IP in the initial runs. \label{tab:summary:os}}
\begin{center}
\begin{tabular}{l|cccc}
\hline
     & \multicolumn{2}{c}{IPs} & \multicolumn{2}{c}{Median} \\
Experiment & Median & Max & time (s) &  \#{\tt  PUT}s \\
\hline
4/4 & 5 & 16 & 2040 & 18   \\
4/24 &  5 & 29 & 732 & 11  \\
7/31 & 5 & 14 & 260 & 23   \\
\hline
Cache=128 & 5 & 17 & 222.2 & 124 \\ 
Cache=64 & 8 & 38 & 51.3 & 100 \\
Cache=32 & 6 & 19 & 58.9 & 45 \\
\hline
% Data in .RData file in Splash-volunteer
\end{tabular}
\end{center}
\end{table}
%

A summary of the results of each run is also shown in Table
\ref{tab:summary:os}, which shows the median number of IPs
intervening in each experiment,  median time needed
to finish the experiment, median number of HTTP {\tt PUT}s per IP. The
first striking result is that in all cases, 50\% of the 
experiments involved 5 or less IPs. This is consistent with previous results
\citep{DBLP:conf/gecco/GuervosG15} which found 6 to be
the expected number of volunteer IPs. The
maximum number of different IPs for each experiment is also in the
same range and of the order of 10, which is also consistent with
prior work and does not vary across the two different batches. 

We will have to analyze differently the median time, since the two
%Pablo: different verbal times ("will have", and before, "is shown"). Homogenize.
batches are solving different problems. In both cases it possesses a
big range of variation, but 50\% of 
the time takes less than several minutes, from around 4 minutes in the
best case to roughly 2/3 of an hour in the worst case. Remarkably
enough, the time is more consistent in the second batch and always
around one minute, in two cases even less, and that happens when the
median number of IPs is higher. It should be noted that while the
first batch of experiments took several days in each case, the second
only lasted for a few hours, with a more continued effort of
publicizing it in social networks. This is specially true in the
%Pablo: removed semicolon before "This". @unintendedbear does not like them :P
case of cache equal to 64, which is noticed by the high number of
volunteers participating in the experiment. The conclusion is that,
in general, the key factor in the time needed to find the solution
is, as expected, the number of volunteers it is able to gather on a
short notice. 

The number of {\tt PUTs}, every one corresponding to 100 generations,
is the algorithmic result. It is relatively unchanged for the first
batch and around 20, that is, 2000 generations or 2000*128 = 256000
evaluations. In this case, an ``unlimited'' cache was used, with all
individuals sent from clients stored until the end of the
experiment. However, we were interested in measuring also the
performance of the algorithm itself by changing the cache size, after
making it limited. As it can be seen in the table, there is a clear
%Pablo: which table?
change in the number of evaluations needed, with smaller cache sizes
producing solutions in less evaluations, until it is for the cache
size = 32 roughly twice as much as with the previous problem, with 40
traps. This is a good result and is also algorithmically consistent
with other results obtained using the same type of problems. Since in
this paper we were interested in leveraging the user's CPU cycles by
improving the algorithm, a good conclusion of this paper is that
having a small pool size helps clients to obtain ``good'' individuals
from the pool, as opposed to any individual that could be obtained
before. Besides, the cache policy deletes the oldest individuals, which
makes those in the pool be {\em current}, helping then newcomers and %Pablo: "then helping newcomers", instead?
any participant obtain the best individuals found in the last part of
the experiment. Besides, a limited cache helps also in cases with a
bigger search space or longer running times when the server simply
crashed due to lack of RAM. 

\begin{figure}[!htb]
\centering
\includegraphics[width=0.32\linewidth]{time-vs-ips-OS-4-4.png}
\includegraphics[width=0.32\linewidth]{time-vs-ips-OS-4-24.png}
\includegraphics[width=0.32\linewidth]{time-vs-ips-OS-7-31.png}
\includegraphics[width=0.32\linewidth]{time-vs-ips-alife-128.png}
\includegraphics[width=0.32\linewidth]{time-vs-ips-alife-64.png}
\includegraphics[width=0.32\linewidth]{time-vs-ips-alife-32.png}
\caption{Duration of experiments vs. number of different IPs (nodes)
  participating in it, with averages per number of IPs and standard deviation shown as
  red dots; in the case there is a single red dot, there was a single
  experiment in which that many computers participated (for instance, 16
  computers in the experiment in the far left or 29 in the middle
  one). 
Shades of blue indicate how many experiments included that many unique IPs,
so lighter shade for a column of dots indicates that a particular number
of computers happened less frequently, while darker shadow means more. 
From left to right and top to bottom experiments 4/4, 4/24 and 7/31,
followed by experiments with 50 traps, cache=128, 64, 32.}
\label{fig:duration}
\end{figure}
%
We will have to analyze experimental data a bit further to find out why
this happens and also if there are some patterns in the three sets of
experiments. An interesting question to ask, for instance, is if 
by adding more computers it makes the experiment take less time. In fact, as
shown in Figure \ref{fig:duration}, the {\em addition} of more computers does
not seem to contribute to decreasing the time needed to finish the
experiment. However, the cause-effect relationship is not clear at
all. It might be the opposite: since experiments take longer to finish
and might in fact be abandoned with no one contributing for some time,
the probability of someone new joining them is higher. In fact,
with experiments taking a few seconds and due to the way the
experiments are announced, it is quite difficult that several
volunteers join in in such a short period of time, even more if we take
into account that volunteers are not {\em carried over} from previous
experiments. 
%
\begin{figure}[!htb]
\centering
\includegraphics[width=0.95\linewidth]{ips-per-minute-cache=64.png}
\caption{Simultaneous IPs every minute of the experiment with cache =
  64. \label{fig:otisdriftwood}}
\end{figure}


That is why we used a more difficult problem in the second batch of
experiments, which is shown in the bottom row of Figure
\ref{fig:duration}. The pattern is remarkably similar, showing a
positive correlation between the time for solving the problem and the
number of computers, at least for cache sizes 128 and 64. However, it
is interesting to observe that, for cache=32, the time needed to find
the solution decreases from one to approximately 4-5 nodes, to then
increase for a higher number of participating computers, distinguished
by IP. The green dot at the bottom is probably an outlier that we
will try to explain later on. This leads us to conclude that a larger
amount of computers might contribute to speed up the solution, if the
time the experiment ideally takes is sufficient, that is, of the order of a
minute, and enough volunteers concur simultaneously. This is also
observed, not so clearly, in the case of cache=64, with an interval of
around 10 IPs obtaining less time than experiments with less or more
IPs, and of the same order, between 10 and 100 seconds. If we look at
the graph that shows the number of IPs or volunteers per minute for
this experiment, shown in Figure \ref{fig:otisdriftwood}, we see that
there are peaks of more than 25 volunteers, and a period of several
hours with a minimum of 6 computers and peaks of more than 10. The
long period after midnight where there is a single volunteer left
masks the success achieved during this set of experiments, from which
we draw two lessons: first, you need a social network influencer to
%Pablo: people know the meaning of "influencer"? Probably yes, but I mention it just in case.
announce your experiments and second, no matter what, do not do any
experiment after midnight. This statement, which might seem tongue in
cheek, in fact, it is a conclusion drawn from the experimental data and
to what extent the social network is an essential part of the
description and performance of the NodIO volunteer computing system. 

\begin{figure}[!htb]
\centering
\includegraphics[width=0.32\linewidth]{time-vs-rank-OS-4-4.png}
\includegraphics[width=0.32\linewidth]{time-vs-rank-OS-4-24.png}
\includegraphics[width=0.32\linewidth]{time-vs-rank-OS-7-31.png}
\includegraphics[width=0.32\linewidth]{time-vs-rank-alife-128.png}
\includegraphics[width=0.32\linewidth]{time-vs-rank-alife-64.png}
\includegraphics[width=0.32\linewidth]{time-vs-rank-alife-32.png}
\caption{Duration of experiments vs. rank of experiments sorted by
  descending duration, with $y$ axis in a
  logarithmic scale. Dot color is related to the number of IPs
  participating in the experiment. From left to right and top to bottom, experiments
  4/4, 4/24 and 7/31 and caches=128, 64, 32.} 
\label{fig:zipf:os}
% First ones made with time-vs-IPs-openshift.R:  
\end{figure}
%
It is also interesting to check the distribution of the experiment
duration, shown in Figure \ref{fig:zipf:os} and which roughly follows
a Zipf's law, with similar distribution along all three runs. The 4/24
run is the most complete and shows an S-shape, which implies an
accumulation of experiments taking similar time and around 100
seconds; this S-shape appears too in the experiments with cache=128
(bottom row, left). The most interesting part is the {\em tail}, which shows how
many experiments took a desirable amount of time, on the order of
10 seconds, and which appears in all three graphs. As it can be seen,
it sharply drops implying there are 
just a few of them, and with diminishing probability as time
decreases. However, since they have a greenish color, implying a low
number of IPs, they might be due to clients {\em carrying over} from
the previous one. This is a characteristic of this implementation
which will be examined later on, but at any rate, if we discard those
experiments that take too much or too little, there is a decreasing
exponential distribution that corresponds to the Zipf's law.

\begin{figure}[!htb]
\centering
\includegraphics[width=0.32\linewidth]{weibull-puts-openshift-4-4.png}
\includegraphics[width=0.32\linewidth]{weibull-puts-openshift-4-24.png}
\includegraphics[width=0.32\linewidth]{weibull-puts-openshift-7-31.png}
\includegraphics[width=0.32\linewidth]{weibull-fit-cache=128.png}
\includegraphics[width=0.32\linewidth]{weibull-fit-cache=64.png}
\includegraphics[width=0.32\linewidth]{weibull-fit-cache=32.png}
\caption{Number of {\tt PUT}s per unique IP and fit to a Weibull
  distribution (in red); axis $x$ shows IPs sorted by descending
  number of PUTs. From left to right and top to bottom, experiments
  4/4, 4/24 and 7/31 and new experiments with cache=128, 64, 32.} 
% Plotted with ../data/plot-zipf-openshift.R
\label{fig:puts:os}
\end{figure}
%
A similar exponential distribution also appears if we rank HTTP {\tt
  PUT}s, equivalents to the number of 
generations divided by 100, or to evaluations divided by 12800,
contributed by every user, which is shown 
in Figure \ref{fig:puts:os}. These results show a Zipf-like behavior,
that is, a power law with a small {\em bump} in the lowest
values. After testing the Generalized Extreme Value distribution and
failing for the new batch of experiments, we have fitted it to a
Weibull distribution \cite{thoman1969inferences} with the resulting
parameters shown in Table \ref{tab:puts:os}. 
%
\begin{table}
\caption{Weibull distribution  parameters of the fit of
  the number of {\tt PUT}s per unique IP. \label{tab:puts:os}}
\begin{center}
\begin{tabular}{l|cc}
\hline
Experiment  &  Scale $\sigma$ & Shape $\xi$ \\
\hline
4/4 &  43.07 $\pm$ 5.80 &  0.57 $\pm$ 0.03 \\
4/24 & 22.97 $\pm$ 1.57 & 0.66 $\pm$  0.02  \\
7/31 &  53.18 $\pm$ 7.77 &  0.54 $\pm$ 0.03   \\
\hline
Cache 128 & 205.28 $\pm$ 32.10 & 0.77 $\pm$ 0.07 \\ 
Cache 64 & 178.99 $\pm$ 36.44 & 0.60 $\pm$ 0.05 \\ 
Cache 32 & 168.15 $\pm$ 49.28 & 0.57 $\pm$ 0.07 \\
\hline
% from plot-zipf-openshift.R and ips-puts-alife.R
\end{tabular}
\end{center}
\end{table}
%
The inverse
Weibull distribution is a special case of the GEV distribution
in those papers, and appears usually in natural sciences and
artificial life, usually related to decay. It has been frequently
fitted to volunteer computing frameworks such as SETI@home
\citep{javadi2009mining}. The model that user behavior follows can be
explained straighforwardly: when users visit the page, it draws their
attention for a limited amount of time. They give it a chance for a
few seconds. If something there amuses them or they can engage in a
conversation about it, they stay for a while longer, otherwise, they
leave. The {\em scale} parameter, which is around 20-40 in the first
batch for the 40 Trap problem and between 160 and 200 in the 60 traps
problem, depends mainly on the maximum number of generations people
leave it running. Since it finishes or stalls after a number of
generations shown in Figure \ref{tab:summary:os}, volunteers just
leave after that. Curiously enough, the scale parameter is roughly
twice the median number of {\tt PUT}s per experiment, showing that, on
average, the most loyal users reload the page twice after finishing or
after seeing the evolution does not progress. This rule of thumb
breaks down with the last experiment, however, which is interesting by
itself, too. 

The slope or shape parameter, on the other hand, indicates the overall
shape of the curve. A value less than 1 indicates a concave (in a
non-algorithmic scale) curve,
with figures closer to one indicating a smaller slope. In all cases
values are between 0.54 and 0.77, independently of the experiment. It
might be the case that this number depends more on the total number of
experiments carried out, with sets with more experiments, both in the
middle, having values between 0.60 and 0.70. The distribution is
remarkably similar which gives us a model of user behavior that is, to
a certain extent, independent of the experiment.  

These experiments show that, as it was proved for other volunteer
computer frameworks and also in the case of games, user engagement
follows a Weibull distribution. This makes engagement the key for
leveraging the performance of the sociotechnical metacomputer and a
way to improve results in the future. 

%---------------------------------------------------------------
\section{Conclusion}
\label{sec:conclusion}

Our intention in this paper was to assess the capabilities of a
sociotechnical system formed by a client-server web-based framework
running a distributed evolutionary algorithms and the volunteers that
participate in the experiment. These volunteers are {\em in the
  cloud}, that is, available as {\em CPU as a service} for the persons
running the experiment. In this paper we have tried to put some figures
on the real size of that {\em cloud} and how it can be used standalone
if there is no alternative, or, if other computing resources are
available, in conjunction with other local or cloud-based methods to
add computing power in a seamless way through the pool that NodIO creates. 

After running the experiment on the 40-trap problem whose running time
could be as low as a few seconds, we switched to another batch of
experiments where we used the 50-trap problem and also to a pool of
limited, and dwindling through the three experiments, size. Since our
initial results indicated that what happened on the screen, a flat
graph with no improvements or the experiment finished, influenced the
amount of time that the users devoted to the experiment, a longer one
could yield different results and, at the same time, result in a big
pool that might either crash the server or return useless individuals
to the volunteers. These new experiments have proved that using the
limited pool is beneficial to finding the solution, since less
evaluations are needed, but also that since the problem is more
difficult, the users stay for longer in the web page, making less
ephemeral the sociotechnical computing system created by the
simulation. 


The second objective of this paper was to model the user behavior in a
first attempt to try and predict performance. As should be expected,
the model depends on the implementation, with contributions following
a Weibull distribution, which reflects the fact
that volunteer computing follows a model quite similar to that found
for games or other online activities. The reverse might be true: if we
want to have returning users for the experiments, it is probable that
we should {\em gamify} the experience so that once they've done it
once, they might do it more times. In the spirit of Open Science, this
gamification might involve computing in real time data such as the one
presented in this paper and showing it in the same page or presenting
user results alongside others.

In general, linking and finding correlations between user choices and
performance is an interesting avenue to explore in the future. Even if
these  experiments were published in a similar way, one
obtained up to five times more total cycles than the one with the least
number of cycles. It is also essential to obtain volunteers as fast and
simultaneously as possible, so it is possible that the features of the
social network in terms of real-time use will also play a big
role; synchronous webs such as Snapchat, which mystifies the writers
of this paper, and Twitter, thanks to its real time nature, might be
better suited than Facebook, LinkedIn or Google plus. Even as it is
difficult to create controlled experiments in this 
area, it is an interesting challenge to explore in the future.

The other area to explore is the algorithmic area itself. Are there
ways to change the evolutionary algorithm, or its visualization, so
that the user has a bigger impact on the result? One of the users in
Twitter even suggested to embed videos so that people spent time
looking at them, but other possible way was to make the user engage
the algorithm by giving him or her buttons to change the mutation rate
when the algorithm is stalled, for instance. If this is combined with
a score board where local performance is compared to other users,
engagement might be increased and thus the performance of the
system. In general there are 
many issues with the evolutionary algorithm implementation itself,
including using different, or adaptive, policies for inserting and
sending individuals to the pool,
using different policies for population initialization, and also the
incorporation of high-speed local resources to the pool to check what
would be the real influence of the volunteer pool to the final
performance. 

Finally, the implementation needs some refinement in terms of
programming and also ease of use. Tools such as Yeoman for generating
easily new experiments might be used, so that the user would have  to
create only a fitness
function, with the rest of the framework wrapped around
automatically. 

All these avenues of experimentation will be done openly following the
Open Science policy of our group, which, in fact, contributes to
establish trust and security between us and volunteers and is an
essential feature of the system. That is why this paper, as well as
the data and processing scripts, are published with a free license in GitHub at
\url{https://github.com/JJ/modeling-volunteer-computing}.

%---------------------------------------------------------------
\section*{Acknowledgments}

This work has been supported in part by TIN2014-56494-C4-3-P (Spanish Ministry of Economy and Competitivity),
PROY-PP2015-06 (Plan Propio 2015 UGR)  % please, remember to include all the current projects   ;)   Thanks!
. We would also like to thank the
anonymous reviewers of previous versions of this paper who have really
helped us to improve 
this paper (and our work) with their suggestions. We are also grateful
to Anna S\'aez de Tejada for her help with the data processing
scripts. We are also grateful to {\tt @otisdriftwood} for his help
gathering users for the new experiments. 


\bibliographystyle{apalike}
\bibliography{volunteer,GA-general,geneura,javascript,ror-js}

\end{document}

%%% Local Variables:
%%% ispell-local-dictionary: "english"
%%% End: