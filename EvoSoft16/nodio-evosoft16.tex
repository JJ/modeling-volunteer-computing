% This is "sig-alternate.tex" V2.1 April 2013
% This file should be compiled with V2.5 of "sig-alternate.cls" May 2012
%
% This example file demonstrates the use of the 'sig-alternate.cls'
% V2.5 LaTeX2e document class file. It is for those submitting
% articles to ACM Conference Proceedings WHO DO NOT WISH TO
% STRICTLY ADHERE TO THE SIGS (PUBS-BOARD-ENDORSED) STYLE.
% The 'sig-alternate.cls' file will produce a similar-looking,
% albeit, 'tighter' paper resulting in, invariably, fewer pages.
%
% ----------------------------------------------------------------------------------------------------------------
% This .tex file (and associated .cls V2.5) produces:
%       1) The Permission Statement
%       2) The Conference (location) Info information
%       3) The Copyright Line with ACM data
%       4) NO page numbers
%
% as against the acm_proc_article-sp.cls file which
% DOES NOT produce 1) thru' 3) above.
%
% Using 'sig-alternate.cls' you have control, however, from within
% the source .tex file, over both the CopyrightYear
% (defaulted to 200X) and the ACM Copyright Data
% (defaulted to X-XXXXX-XX-X/XX/XX).
% e.g.
% \CopyrightYear{2007} will cause 2007 to appear in the copyright line.
% \crdata{0-12345-67-8/90/12} will cause 0-12345-67-8/90/12 to appear in the copyright line.
%
% ---------------------------------------------------------------------------------------------------------------
% This .tex source is an example which *does* use
% the .bib file (from which the .bbl file % is produced).
% REMEMBER HOWEVER: After having produced the .bbl file,
% and prior to final submission, you *NEED* to 'insert'
% your .bbl file into your source .tex file so as to provide
% ONE 'self-contained' source file.
%
% ================= IF YOU HAVE QUESTIONS =======================
% Questions regarding the SIGS styles, SIGS policies and
% procedures, Conferences etc. should be sent to
% Adrienne Griscti (griscti@acm.org)
%
% Technical questions _only_ to
% Gerald Murray (murray@hq.acm.org)
% ===============================================================
%
% For tracking purposes - this is V2.0 - May 2012

\documentclass{sig-alternate}

\graphicspath{{../img/}}

% *** SPECIALIZED LIST PACKAGES ***
%
\usepackage{algorithmic}

\usepackage{array}

% *** PDF, URL AND HYPERLINK PACKAGES ***
%
\usepackage{url}

\begin{document}

% Copyright
\setcopyright{acmcopyright}
%\setcopyright{acmlicensed}
%\setcopyright{rightsretained}
%\setcopyright{usgov}
%\setcopyright{usgovmixed}
%\setcopyright{cagov}
%\setcopyright{cagovmixed}

%Conference
\conferenceinfo{GECCO '16}{Denver, CO, USA}
% --- End of Author Metadata ---

\title{NodIO: a framework for pool-based evolutionary computation}
%
% You need the command \numberofauthors to handle the 'placement
% and alignment' of the authors beneath the title.
%
% For aesthetic reasons, we recommend 'three authors at a time'
% i.e. three 'name/affiliation blocks' be placed beneath the title.
%
% NOTE: You are NOT restricted in how many 'rows' of
% "name/affiliations" may appear. We just ask that you restrict
% the number of 'columns' to three.
%
% Because of the available 'opening page real-estate'
% we ask you to refrain from putting more than six authors
% (two rows with three columns) beneath the article title.
% More than six makes the first-page appear very cluttered indeed.
%
% Use the \alignauthor commands to handle the names
% and affiliations for an 'aesthetic maximum' of six authors.
% Add names, affiliations, addresses for
% the seventh etc. author(s) as the argument for the
% \additionalauthors command.
% These 'additional authors' will be output/set for you
% without further effort on your part as the last section in
% the body of your article BEFORE References or any Appendices.

\numberofauthors{5} %  in this sample file, there are a *total*
% of EIGHT authors. SIX appear on the 'first-page' (for formatting
% reasons) and the remaining two appear in the \additionalauthors section.
%
\author{
\alignauthor
Juan-J.~Merelo, Pedro A. Castillo,Pablo Garc\'ia-S\'anchez,Paloma de las Cuevas\\
\affaddr{Dept. of Computer Architecture and Technology
  and CITIC, University of Granada, Granada, Spain} \\
\email{(jmerelo|pacv|pgarcia|palomacd)@ugr.es}
\alignauthor
Mario Garc\'ia-Valdez,\affaddr{ Dept. of Graduate Studies, Instituto
Tecnológico de Tijuana, Tijuana, M\'exico}\\
}

\maketitle

\begin{abstract}

\end{abstract}

\keywords{Volunteer computing, distributed computing, cloud computing}


%---------------------------------------------------------------
\section{Introduction}


So, the rest of the paper is organized as follows: next we present the
state of the art in web-based distributed
computational systems along with attempts to predict and model their
behavior in Section \ref{sec:soa}. A brief presentation of the
methodology used and the results of the experiments are presented in
%Pablo: mention the problems used in the first experiments
Section \ref{sec:exp1}, including new results using web workers on the
same problem and on a new problem, a modified Rastrigin's function. 
%Pablo: Rastrigin is not the owner of the function
%Mario: In many places is called  Rastrigin's function because it was proposed by him.
Finally, conclusions and future lines of work are presented in Section
\ref{sec:conclusion}. 

%---------------------------------------------------------------
\section{State of the art}
\label{sec:soa}



\section{The NodIO framework and API}
\label{sec:exp1}


%---------------------------------------------------------------
\section{Conclusions and future work}
\label{sec:conclusion}


%---------------------------------------------------------------
\section*{Acknowledgment}
 
This work has been supported in part by
TIN2014-56494-C4-3-P (Spanish Ministry of Economy and Competitivity), PROY-PP2015-06 (Plan Propio 2015 UGR), 
SPIP2014-01437 (Direcci{\'o}n General de Tr{\'a}fico) and PYR-2014-17
GENIL project (CEI-BIOTIC Granada). Additional support was recieved by
Projects 5622.15-P (ITM) and  PROINNOVA 2015: 220590 (CONACYT).
We would also like to thank the
anonymous reviewers of previous versions of this paper who have really
helped us to improve 
this paper (and our work) with their suggestions. We are also grateful
to Anna S\'aez de Tejada for her help with the data processing scripts.

\bibliographystyle{abbrv}
\bibliography{geneura,volunteer,javascript,ror-js,GA-general}

\end{document}

%%% Local Variables:
%%% ispell-local-dictionary: "english"
%%% End:
