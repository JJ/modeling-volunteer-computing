\documentclass[runningheads,a4paper]{llncs}

\usepackage[latin1]{inputenc}
\usepackage{amssymb}
\usepackage{amsmath}
\setcounter{tocdepth}{3}
\usepackage{graphicx}
\usepackage{multirow}
\usepackage{rotating}
\usepackage{subfigure}
%\usepackage{subfig}
\usepackage{url}
\usepackage{caption}

\newcommand{\keywords}[1]{\par\addvspace\baselineskip
\noindent\keywordname\enspace\ignorespaces#1}

\providecommand{\tabularnewline}{\\}

\begin{document}

\mainmatter  % start of an individual contribution

% first the title is needed
\title{Title TBD}

% a short form should be given in case it is too long for the running head
\titlerunning{TBD}

% the name(s) of the author(s) follow(s) next
%
% NB: Chinese authors should write their first names(s) in front of
% their surnames. This ensures that the names appear correctly in
% the running heads and the author index.
%
\author{Several A. Uthors}
%
%\authorrunning{A. Fern\'andez-Ares et al.}
% (feature abused for this document to repeat the title also on left hand pages)

% the affiliations are given next; don't give your e-mail address
% unless you accept that it will be published
%\institute{Dept. of Computer Architecture and Technology, University
%of Granada, Spain}
\institute{Dept. of Various Stuff, Imaginary Country}

%
% NB: a more complex sample for affiliations and the mapping to the
% corresponding authors can be found in the file "llncs.dem"
% (search for the string "\mainmatter" where a contribution starts).
% "llncs.dem" accompanies the document class "llncs.cls".
%




\maketitle

%
%%%%%%%%%%%%%%%%%%%%%%%%%%%%%%%   ABSTRACT   %%%%%%%%%%%%%%%%%%%%%%%%%%%%%%%
%
\begin{abstract}
\keywords{Videogames, RTS, evolutionary algorithms, termination criteria, noisy fitness}
\end{abstract}

%
%%%%%%%%%%%%%%%%%%%%%%%%%%%%%%%   INTRODUCTION   %%%%%%%%%%%%%%%%%%%%%%%%%%%%%%%
%
\section{Introduction}


The rest of the paper is organized as follows. Next, section \ref{sec:SoA} establishes the state of the art related to termination conditions for evolutionary algorithms in which the optimum is not known and/or is noisy. Following, section \ref{sec:met} presents the set of termination conditions which have been checked in this paper and how they contribute to prove which one has better behaviour in this context; we also sketch the problem we have used in this particular paper, evolution of bot for Planet Wars. Results are presented in Section \ref{sec:met}, followed by the conclusions and a discussion of the implication of the obtained results.

%%%%%%%%%%%%%%%%%%%%%%%%%%%%%%  STATE OF THE ART  %%%%%%%%%%%%%%%%%%%%%%%%%%%%%
%
\section{State of the Art}
\label{sec:SoA}



%
%%%%%%%%%%%%%%%%%%%%%%%%%%%%%%  METHODOLOGY  %%%%%%%%%%%%%%%%%%%%%%%%%%%%%
%
\section{Methodology and experimental setup}
\label{sec:met}

%
\section{Experiments and Results}
\label{sec:res}


%%%%%%%%%%%%%%%%%%%%%%%%%%%%%%%%  CONCLUSIONS  %%%%%%%%%%%%%%%%%%%%%%%%%%%%%%%%
%
\section{Conclusions}

\section*{Acknowledgments}

Hidden for double-blind review
% \scriptsize{This work has been supported in part by SIPESCA (Programa Operativo FEDER de Andaluc\'ia 2007-2013), SPIP2014-01437 (Direcci\'on General de Tr\'afico), PRY142/14 (Fundaci\'on P\'ublica Andaluza Centro de Estudios Andaluces en la IX Convocatoria de Proyectos de Investigaci\'on) and PYR-2014-17 GENIL project (CEI-BIOTIC Granada).}


\bibliographystyle{splncs}
\bibliography{geneura,volunteer}

\end{document}
