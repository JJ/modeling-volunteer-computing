\documentclass[preprint]{elsarticle}
\biboptions{round, numbers}
\usepackage[latin1]{inputenc}
%\usepackage[T1]{fontenc}
%\usepackage{textcomp}
\usepackage{graphicx}
\usepackage{color}
%\usepackage{setspace}
\usepackage{url}
\usepackage[english]{babel}

\begin{document}

%%%%%%%%%%%%%%%%%%%%%%%%%%%%%%%   TITLE   %%%%%%%%%%%%%%%%%%%%%%%%%%%%%%%

\title{Volunteers in the Clouds: an Architecture for Low Cost and
  Potentially Massive Distributed Evolutionary Computation: Response to Reviewers\' comments}

\noindent
Dear Sirs,\\

We really appreciate the opportunity you have given us for improving
our work. Following the reviewers' requests and suggestions we have
deeply changed the contents, and  thus, the quality of the paper. 

In the next paragraphs you can find these comments and our responses
to every one, along with the changes done.\\ 

\noindent
Yours sincerely,\\
The authors.

\section{Response to Reviewer \#1}

\begin{quote}
The main contribution of this paper is the proposal of a Javascript based distributed system intended to
run evolutionary algorithms according to the volunteer computing scheme. The idea is that user can
contribute to the computation by using only a browser, without needing to install any plugin nor a Java
virtual machine.
The volunteer computing model appeared some years ago and it has been used in many fields. The
model is effective to solve problems demanding high computational capabilities, and at this point is where
this paper proposal introduces some concerns. The choice of Javascript code running on browsers has
some advantages, as is pointed out in the paper, but some important drawbacks. Given that the
motivation of these kinds of distributed models is to solve complex problems, using Javascript doesnÕt
appear a right choice. ItÕs true that is a very commonly used programming language, but in the context of
Web applications, not in high performance ones. Apart from the fact of being an interpreted language
designed for the Web, it is difficult to think on a computer scientist writing a complex math code in
Javascrip. If the authors would have used as a case study a complex optimization problem then they
could have provided numerical data to sustain their claims; however, the selected problem (which should
be formulated in the paper) seems a rather academic one.
\end{quote}

\begin{quote}
The paper organization is not clear. Some comments about this:
- The title ÒVolunteers in the Clouds: an Architecture for Low Cost and Potentially Massive Distributed
Evolutionary ComputationÓ seems to suggest that the volunteers run in the cloud, but that is not the case:
the server run􀀀 in the cloud and the volunteers do in the browsers of common PCs and laptops.
- The paper abstract does not properly summarize the paper contents. It says that ÒThat is why in this
paper we present NodIO, a client-server architecture for distributed evolutionary algorithms !Ó, but there
are two versions of the architecture, as stated later in the conclusions: ÒIn this paper two versions of a
client-server architecture for volunteer and distributed evolutionary !Ó. That second version is neither
commented in the introduction.
\end{quote}

\begin{quote}
presented in Section III, followed by the experiments and obtained results in Section IV.Ó, but Section IV
does not only contains experimental result but also the full description of the second version of the
architecture.
Summarizing the review, the authors should provide a convincing example showing the advantages of
their proposal. As presented, the paper is a proof of concept of a system that seems difficult to be used
by the researchers needing high computational resources to run an evolutionary algorithm to solve a realworld
problem.
\end{quote}

\section{Response to Reviewer \#1}

\begin{quote}
Contribution:
This paper describes a client-server architecture for distributed evolutionary algorithms named NodIO.
The implementation of the architecture is written entirely in JavaScript, in order to support persistent,
asynchronous, distributed evolutionary algorithms running in the browser.
Additionally, the paper presents the results of several experiments performed to validate the proposal and
to measure the performance improvement generated by the massive parallel computing.
I think that the contribution is quite relevant, since it provides a computation platform that is flexible,
extensible and potentially massive. Personally, I consider it one of the fresher and most promising
proposals in the context of metaheuristic optimization frameworks lately. Hence, our global
recommendation of acceptance with minor changes.
\end{quote}

\begin{quote}
However, the paper still has plenty of room for improvement in three main areas: the experimentation, the
organization and description of the platform, and the presentation and writing style.
Regarding the experimentation, I miss:
• a section discussing the threats to validity of the experiments
• an experiment with a more difficult problem (please use the well-known benchmark problems
described in literature).
• A comparison in similar conditions of the two implementations of the architecture (the algorithms
used in the experiments are slightly different, but this could lead to dramatic changes in
performance, and consequently, invalidate the comparison performed in the paper). This would
enable a measurement of the actual improvement in performance obtained by the use of web
workers (not that generated due to a different algorithm)
\end{quote}

\begin{quote}
Regarding the organization of the paper, some important details regarding the improved framework
design named is described in the experimentation section. I would reorganize the contents to describe the
improved in detail in section 3, and focus on the experiments, its results and the discussion in section 4.
Another issue is that some details are described twice, for instance, the advantages of the use of web
workers. Fixing this issue can reduce the paper size and improve its readability.
Regarding the presentation:
• Figures should be readable even in gray-scale, so do not use red and blue to denote different
elements, use a different shade or texture.
• The axis labels and legends in most of the figures are unreadable, please increase its font size.
• There are some typos and phrases that need rewriting.
\end{quote}

\section{Response to Reviewer \#3}

\section{Response to Reviewer \#4}

\begin{quote}
This paper presents a very challenging proposal. Namely, authors introduce a novel client-server
architecture for low cost and “potentially” massive distributed evolutionary algorithms. The main novelty
turns up from the fact that the architecture is developed in JavaScript.
However, JavaScript is well-known because of its lack of capability to deal with high performance
applications like the ones that are addressed in this paper. Thus, the use of JavaScript makes hard to
believe in the possibility of “potentially” massive computation as sketched by the authors.
The paper is well written and organized. State of the art is quite complete. Nevertheless, the experimental
section must be carefully checked and extended in order to fully validate the goodness of the proposed
architecture. At least, authors should show the behavior of the architecture in a real-world problem
demanding high computational performance.
Finally, English must be double-checked in order to correct minor typos such as “in the sense than[that]”
(in page 2).
\end{quote}



\end{document}
